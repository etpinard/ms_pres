%
% Section 1: Introduction
%
%   # of slides: 10
%
% -------------------------------------------------------------------------------
\section{Introduction}
\label{sec:intro}

\begin{frame}
\frametitle{Why Study Temperature Variability?}

\begin{itemize}

\item {Recent summertime heat waves / droughts:}

\end{itemize}

\begin{center}

\tikzfig{intro_fig}{width=0.75\linewidth}

\end{center}

\begin{tikzpicture}[overlay]

\coordinate (p2003) at ($(intro_fig.west)+(1.75,0.5)$);
\coordinate (p2010) at ($(intro_fig.west)+(-0.5,-2)$);
\coordinate (p2012) at ($(intro_fig.east)+(0.5,-2)$);

\node[color=colone,font=\bf] at (p2003) {2003};
\node[color=colone,font=\bf] at (p2010) {2010};
\node[color=colone,font=\bf] at (p2012) {2012};

\end{tikzpicture}

\end{frame}

\begin{frame}
\frametitle{Temperature variability projections}

\begin{columns}

  \begin{column}{0.35\linewidth}

  \vspace*{-2em}
  \begin{itemize}

  \item \citet{schar04}: \textcolorbf{coltwo}{increasing \COtwo} could lead to
  \textcolorbf{coltwo}{increasing summertime $T$ variability} in some locations.

  \item \textcolorbf{colone}{Left panel:} \newline \vspace*{-1em}
  $\barr{T}_{2071-2100}-\barr{T}_{1961-1990}$ \\[1.5em]

  \textcolorbf{colone}{Right panel:} \newline \vspace*{-0.5em}
  $\dfrac{\sigma_{\!T,2071-2100}-\sigma_{\!T,1961-1990}}{\sigma_{\!T,1961-1990}}$ 

  \end{itemize}

  \end{column}
  
  \begin{column}{0.6\linewidth}

  \vspace*{-0.6em}
  \hspace*{0.7em}
  \tikzfig{schar04_map}{trim=0cm 2.4cm 0.1cm 0cm,clip,width=0.98\linewidth}

%  \vspace*{-1.8em}
%  \tikzfig{schar04_pdf}{trim=0cm 0cm 0.1cm 0cm,clip,width=\linewidth}

  \end{column}

\end{columns}

\end{frame}

\begin{frame}
\frametitle{Role of soil moisture / {\normalsize land-atmosphere interactions}}

\vspace*{-2.5cm}
\begin{itemize}

\item \citet{sene06_coupling}: 4 GCM simulations:

\end{itemize}

\begin{center}

\tikzfig{sene06_sig_T}{trim=0cm 0.4cm 7.5cm 0cm,clip,width=0.68\linewidth}

\vspace*{-4cm} 
\tikzitemMark
\end{center}

\begin{tikzpicture}[overlay]

\coordinate (title) at ($(sene06_sig_T.north)+(0,-0.1)$);
\coordinate (units) at ($(sene06_sig_T.east)+(0.3,-1.7)$);
\coordinate (label1) at ($(sene06_sig_T.west)+(-0.5,2)$);
\coordinate (label2) at ($(sene06_sig_T.west)+(-0.5,-1)$);
\coordinate (label3) at ($(sene06_sig_T.south)+(-1.8,-0.1)$);
\coordinate (label4) at ($(sene06_sig_T.south)+(1.8,-0.1)$);

\node[color=colone,font=\bf] at (title) {Standard deviation of $T$ (JJA)};
\node at (units) {[$\degC$]};
\node[color=colone,align=left,font={\small \bf}] at (label1) 
  {evolving \\ soil moisture};
\node[color=colone,align=left,font={\small \bf}] at (label2) 
  {constant \\ soil moisture};
\node[font={\small \bf}] at (label3) {1960-1989};
\node[font={\small \bf}] at (label4) {2070-2099};

\uncover<2->{%
%
\node[inner sep=10pt,fill=colbg,xshift=-1em] at (item)
{\textcolor{coltwo}{\scalebox{1.3}{$\Longrightarrow$}}\quad 
 \textcolorbf{coltwo}{Soil moisture is related to the $\Var(T)$ trends}} ;}


\end{tikzpicture}

\end{frame}

\begin{frame}
\frametitle{Hold on! Are GCMs reliable?}

\begin{itemize}

\item CMIP5 ensemble mean of summer (JJA in NH, DJF in SH) \newline 
surface temperature variance [1969-1999].

\end{itemize}

\begin{center}

\tikzfig{cmip5_Var_T_bias_Eq}{width=0.9\textwidth}

\vspace*{1em}
\tikzitemMark

\end{center}

\begin{tikzpicture}[overlay]

\tikzitem[<1>]{%
%
Plotted as \enskip $\dfrac{\Var(T)\subt{CMIP5}}{\Var(T\subt{observations})}$
\enskip,\quad \parbox{0.42\textwidth}{\small %
%
$T\equiv$ 2-meter temp.\\ 
$T\subt{observations}$ : U. of Delaware data.
%
}}
%(65 runs, 25 models, 12 institutes).

\tikzitemImply[<2->]{%
%
\textcolorbf{coltwo}{Wide-spread overestimation of $\Var(T)$: 25\% to 100\%.} }

\uncover<2->{%
%
\node[draw,highlight,circle,minimum size=1.3cm] 
  at ($(cmip5_Var_T_bias_Eq.east)+(-0.7,1)$) {}; }
%
%\node[draw,highlight,circle,minimum size=1.3cm] 
%  at ($(cmip5_Var_T_bias.center)+(-0.5,1)$) {}; 
%%  
%\node[draw,highlight,circle,minimum size=1.3cm] 
%  at ($(cmip5_Var_T_bias.center)+(1,1)$) {}; }

%% also in other studies such as \citet{vidale07}

\end{tikzpicture}


\end{frame}

\begin{frame}
\frametitle{Questions:}

\vspace*{-1em}

\begin{itemize}

\boxitem{\centering%
%
Could the \textcolorbf{colone}{same mechanisms} that produce the \\[0.5em] 
\textcolorbf{colone}{\twentyfirst century} summertime $\Var(T)$
\textcolorbf{colone}{trends} in the GCMs \\[0.5em]
be \textcolorbf{colone}{responsible for} the \\[0.5em] 
\textcolorbf{colone}{\twentyth century} summertime $\Var(T)$ 
\textcolorbf{colone}{biases} in the GCMs ? }

\end{itemize}

\vspace*{1em}

\uncover<2->{
%
But first:

\begin{columns}
\begin{column}{0.3\linewidth}

\hspace*{1cm} \textcolorbf{colone}{How are}

\end{column}
\begin{column}{0.4\linewidth}
\begin{itemize}

\itemDiamond soil moisture,
\itemDiamond land-atmosphere \\ \quad interactions, 
\itemDiamond surface temperature \\ \quad variability

\end{itemize}
\end{column}
\begin{column}{0.3\linewidth}

\quad \textcolorbf{colone}{related ?}

\end{column}
\end{columns}
%
}

\end{frame}

\begin{frame}
\frametitle{Mechanisms: Surface Energy Budget}

% define colors, style for equation 
\colorlet{sfc}{black}
\colorlet{F}{Crimson}
\colorlet{E}{MidnightBlue}
\colorlet{Hs}{Magenta}
\colorlet{Flu}{DarkOrange}
\colorlet{G}{Green}
\tikzset{eq/.style={rounded corners=2pt, minimum height=5ex, minimum width=5ex}}

\begin{equation}
%
  \derv{}{t}\,\big(\pt C\subt{eff} \pt T \pt\big) \;\;=\;\;
%     
  \tikzeqMark[fill=F!40,eq]{F}{\Fsw\down - \Fsw\up + \Flw\down} \; - \;  
%
	\tikzeqMark[fill=E!40,eq]{E}{L\,E} \; - \;  
%
	\tikzeqMark[fill=Hs!40,eq]{Hs}{H_s} \; - \;  
%
	\tikzeqMark[fill=Flu!40,eq]{Flu}{\Flu} \; - \;
%
	\tikzeqMark[fill=G!40,eq]{G}{G} 
%
\end{equation}

\vspace*{1em}
\begin{itemize}

\item[]<2->
%
\textcolor{F}{Radiation Forcing} \tikzmark{item-F} \\
\quad $\equiv\, F$

\item[]<3->
%
\textcolor{E}{Evapotranspiration} \tikzmark{item-E}

\item[]<3->
%
\textcolor{Hs}{Sensible Heat Flux} \tikzmark{item-Hs}

\item[]<4->
%
\textcolor{Flu}{Upwelling Longwave Radiation} \tikzmark{item-Flu}

\item[]<4->
%
\textcolor{G}{Ground Heat Flux} \tikzmark{item-G}

\end{itemize}

\vspace*{1em}
%
{\small%
%
$C\subt{eff}$ : specific heat capacity of land surface, \newline
%
$L$ : specific latent heat of vaporization.}

\begin{tikzpicture}[overlay]

	\path[->]<2-> (item-F) edge [bend right] (F);
	\path[->]<3-> (item-E) edge [bend right] (E);
	\path[->]<3-> (item-Hs) edge [out=0, in=-45] (Hs);
	\path[->]<4-> (item-Flu) edge [out=0, in=-45] (Flu);
	\path[->]<4-> (item-G) edge [out=0, in=-45] (G);

\end{tikzpicture}

%% input from figures/diagrams
\tikz{\coordinate (O) at ($(item-G)+(4.5,-0.8)$);}
\ifdraft{}{\begin{tikzpicture}[overlay]
%
% Diagram to complement the 'surface energy budget' slide
%
% ** must define the coordinate (O) before \input call **
% ** as well as surface energy budget colors **
%
% requires the following tikz libraries:
%   arrows, calc, decorations.pathmorphing, shapes
% -------------------------------------------------------------------------------

%% style definitions
\tikzset{%
  sfc/.style={thick,color=sfc},
	rad/.style={very thick,->,
              decorate,decoration={snake,segment length=17pt}},
	flux/.style={very thick,->,
               scale=1,midway,left,
               decorate,decoration={coil,amplitude=4pt,segment length=7pt}},
	ref/.style={very thick,->},
  node-words/.style={scale=0.8,color=black},
}
% ------------------------------------------------------------------------------

%% coordinate definitions
\coordinate (O') at ($(O)+(3.5,0)$);

\coordinate (sw-down) at ($(O)+(-0.2,1.5)$);
\coordinate (sw-down') at ($(O)+(0.2,0.1)$);
	
\coordinate (sw-up) at ($(O)+(0.3,0.1)$);
\coordinate (sw-up') at ($(O)+(0.7,1.5)$);
	
\coordinate (lw-down) at ($(O)+(1,1.5)$);
\coordinate (lw-down') at ($(O)+(1,0.1)$);
	
\coordinate (e) at ($(O)+(2,0.1)$);
\coordinate (e') at ($(O)+(2,1.5)$);
	
\coordinate (hs) at ($(O)+(2.5,0.1)$);
\coordinate (hs') at ($(O)+(2.5,1.5)$); 

\coordinate (lw-up) at ($(O)+(3.2,0.1)$);
\coordinate (lw-up') at ($(O)+(3.2,1.5)$);

\coordinate (g) at ($(O)+(2.8,-0.1)$);
\coordinate (g') at ($(O)+(2.8,-0.7)$);
% -------------------------------------------------------------------------------

%% drawing commands

% draw surface line
\uncover<2->{%
%
\draw[sfc] (O) -- (O'); }

% draw F arrows
\uncover<2->{%
%
\draw[rad,color=F] (sw-down) -- (sw-down') 
  node[node-words,midway,xshift=-2ex] {$\Fsw\down$}; 
\draw[ref,color=F] (sw-up) -- (sw-up') 
  node[node-words,yshift=0ex,xshift=-2.5ex] {$\Fsw\up$}; 
\draw[rad,color=F] (lw-down) -- (lw-down') 
  node[node-words,midway,xshift=2.5ex] {$\Flw\down$}; }

% draw E and H_s arrows
\uncover<3->{%
%
\draw[flux,color=E] (e) -- (e') 
  node[node-words,yshift=1.5ex,xshift=1.7ex] {$L\,E$}; 
\draw[flux,color=Hs] (hs) -- (hs') 
  node[node-words,yshift=1.5ex,xshift=1.7ex] {$H_s$}; }

% draw Flu and G arrows
\uncover<4->{%
%
\draw[rad,color=Flu] (lw-up) -- (lw-up') 
  node[node-words,yshift=1.5ex,xshift=0.5ex] {$\Flu$}; 
\draw[rad,color=G] (g) -- (g') 
  node[node-words,midway,xshift=-1.5ex] {$G$}; }

% -------------------------------------------------------------------------------

\end{tikzpicture}
}
%\begin{tikzpicture}[overlay]
%
% Diagram to complement the 'surface energy budget' slide
%
% ** must define the coordinate (O) before \input call **
% ** as well as surface energy budget colors **
%
% requires the following tikz libraries:
%   arrows, calc, decorations.pathmorphing, shapes
% -------------------------------------------------------------------------------

%% style definitions
\tikzset{%
  sfc/.style={thick,color=sfc},
	rad/.style={very thick,->,
              decorate,decoration={snake,segment length=17pt}},
	flux/.style={very thick,->,
               scale=1,midway,left,
               decorate,decoration={coil,amplitude=4pt,segment length=7pt}},
	ref/.style={very thick,->},
  node-words/.style={scale=0.8,color=black},
}
% ------------------------------------------------------------------------------

%% coordinate definitions
\coordinate (O') at ($(O)+(3.5,0)$);

\coordinate (sw-down) at ($(O)+(-0.2,1.5)$);
\coordinate (sw-down') at ($(O)+(0.2,0.1)$);
	
\coordinate (sw-up) at ($(O)+(0.3,0.1)$);
\coordinate (sw-up') at ($(O)+(0.7,1.5)$);
	
\coordinate (lw-down) at ($(O)+(1,1.5)$);
\coordinate (lw-down') at ($(O)+(1,0.1)$);
	
\coordinate (e) at ($(O)+(2,0.1)$);
\coordinate (e') at ($(O)+(2,1.5)$);
	
\coordinate (hs) at ($(O)+(2.5,0.1)$);
\coordinate (hs') at ($(O)+(2.5,1.5)$); 

\coordinate (lw-up) at ($(O)+(3.2,0.1)$);
\coordinate (lw-up') at ($(O)+(3.2,1.5)$);

\coordinate (g) at ($(O)+(2.8,-0.1)$);
\coordinate (g') at ($(O)+(2.8,-0.7)$);
% -------------------------------------------------------------------------------

%% drawing commands

% draw surface line
\uncover<2->{%
%
\draw[sfc] (O) -- (O'); }

% draw F arrows
\uncover<2->{%
%
\draw[rad,color=F] (sw-down) -- (sw-down') 
  node[node-words,midway,xshift=-2ex] {$\Fsw\down$}; 
\draw[ref,color=F] (sw-up) -- (sw-up') 
  node[node-words,yshift=0ex,xshift=-2.5ex] {$\Fsw\up$}; 
\draw[rad,color=F] (lw-down) -- (lw-down') 
  node[node-words,midway,xshift=2.5ex] {$\Flw\down$}; }

% draw E and H_s arrows
\uncover<3->{%
%
\draw[flux,color=E] (e) -- (e') 
  node[node-words,yshift=1.5ex,xshift=1.7ex] {$L\,E$}; 
\draw[flux,color=Hs] (hs) -- (hs') 
  node[node-words,yshift=1.5ex,xshift=1.7ex] {$H_s$}; }

% draw Flu and G arrows
\uncover<4->{%
%
\draw[rad,color=Flu] (lw-up) -- (lw-up') 
  node[node-words,yshift=1.5ex,xshift=0.5ex] {$\Flu$}; 
\draw[rad,color=G] (g) -- (g') 
  node[node-words,midway,xshift=-1.5ex] {$G$}; }

% -------------------------------------------------------------------------------

\end{tikzpicture}


\end{frame}

\begin{frame}
\frametitle{Mechanisms: Koster Diagram}

\vspace*{-1em}
\begin{center}

\tikzfig{koster}{width=0.9\linewidth}

\vspace*{2.5em}
\tikzitemMark

\end{center}

\begin{tikzpicture}[overlay]

\tikzitem[<1>]{\parbox{0.7\textwidth}{\centering%
%
Simplified framework for understanding $E$ \\ 
on seasonal time scales as a function of: \\[0.5em]
soil moisture ($m$\,)\, and\, $F\subt{net}\equiv F{-}\Flu$ \\[0.5em]
\hfill \citep{koster09_droughts}.} }

\tikzitem[<2>]{%
%
\textcolorbf{coltwo}{Problem:} \enskip
no explicit link to surface temperature. }

\end{tikzpicture}

\end{frame}

\begin{frame}
\frametitle{Mechanisms: Feedback Loops}

\vspace*{-2em}
\begin{columns}

\begin{column}{0.5\linewidth}

\hspace*{-1em}
\tikzfig{sene10_loops1}{trim=0cm 3.1cm 0cm 0cm,clip,width=1.2\textwidth}

\end{column}

\begin{column}{0.5\linewidth}

\hspace*{-4.5em}
\tikzfig{sene10_loops2}{trim=0cm 2.8cm 0cm 0cm,clip,width=1.4\textwidth}

\end{column}

\end{columns}

\centering
\vspace*{4em}
\tikzitemMark

\begin{tikzpicture}[overlay]

% define for cross out
\tikzset{crossout/.style={% 
  draw,line width=1em,cross out,
  minimum width=12em,minimum height=12em,color=coltwo}}

% define coordinate
\coordinate (sene10-middle) at ($(sene10_loops1) !0.5! (sene10_loops2)$);

\tikzitem[<1>]{\parbox{0.7\textwidth}{%
%
\centering
Land-atmosphere interactions are thought \\ to spawn feedbacks loops \\[1ex]
\citep{sene10}.} }

\tikzitem[<2>]{%
%
\textcolorbf{coltwo}{Problem:} \quad \parbox{0.7\textwidth}{%
too complicated, \\[1ex] must run GCM which have large $\Var(T)$ errors.} }

\uncover<2>{%
%
\node[crossout] at (sene10-middle) {}; }


\end{tikzpicture}

\end{frame}

\begin{frame}
\frametitle{Mechanisms: Something better?}

\vspace*{-1em}

% top item
\begin{itemize}

%\boxitem{% Does not work: leading item is misplaced
\item
%
\textcolorbf{colone}{Make link between} \quad
%
\parbox{0.48\linewidth}{
%
\begin{itemize}
\itemDiamond soil moisture,
\itemDiamond land-atmosphere interactions, 
\itemDiamond surface temperature variability
\end{itemize} }
%
\quad \textcolorbf{colone}{explicit}

\end{itemize}

%\vspace*{1em}

% two figures (koster and sene10_loops1)
\begin{columns}
\begin{column}{0.4\linewidth}

\uncover<1->{%
%
\hspace*{-2em}
\tikzfig{koster}{width=1.2\linewidth} }

\end{column}
\hfill
\begin{column}{0.4\linewidth}

\uncover<1->{%
%
\tikzfig{sene10_loops1}{trim=0cm 3.05cm 0cm 0cm,clip,width=0.9\textwidth} }

\end{column}
\end{columns}

\vspace*{2em}

% box item (at bottom)
\begin{itemize}

\boxitem[0.88][<2->]{\centering\textcolorbf{colone}{%
%
A \emph{toy model}\, for surface temperature variability.}}
% 'summertime does not fit'

\end{itemize}

% overlay
\begin{tikzpicture}[overlay]

\coordinate (left) at ($(koster.south west)+(0.4,-0.4)$);
\coordinate (right) at ($(sene10_loops1.south east)+(0,-0.1)$);
\coordinate (toy-model) at ($(koster.east) !0.5! (sene10_loops1.west)$);

\uncover<1->{%
%
\draw[->,very thick] (left) -- (right);
\node[xshift=7.8em,yshift=-2ex] at (left) {conceptual}; 
\node[xshift=-5.8em,yshift=-2ex] at (right) {complex}; 
\draw[very thick] ($(left)+(2.7,-0.1)$) -- ($(left)+(2.7,0.1)$); 
\draw[very thick] ($(right)+(-2,-0.1)$) -- ($(right)+(-2,0.1)$); }

\uncover<2->{%
%
\draw[very thick] ($(toy-model)+(0,-1.78)$) -- ($(toy-model)+(0,-1.58)$); 
\node[circle,draw=black,thick,text=colone,font={\bf \Large}] at (toy-model)
  {us!}; }

\end{tikzpicture}

\end{frame}

\begin{frame}
\frametitle{Outline}

\begin{itemize}

\boxitem{\textcolorbf{colone}{\centering%
%
Let's investigate how soil moisture and $\Var(T)$ are related.}}

\end{itemize}

\vspace*{1em}

\hfill\parbox{0.8\textwidth}{%
%
\begin{itemize}

\item \ref{sec:intro}.\; \nameref{sec:intro} 
  \qquad \scalebox{1.5}{\textcolor{coltwo}{$\checkmark$}}

\item \ref{sec:data}.\; \nameref{sec:data} 

\item \ref{sec:toy_model}.\; \nameref{sec:toy_model} 

\item \ref{sec:interpret}.\; \nameref{sec:interpret} 

\item \ref{sec:apps}.\; \nameref{sec:apps} 

\item \ref{sec:conclu}.\; \nameref{sec:conclu}  \\[1.5em]

\end{itemize}
%
}



\end{frame}

% -------------------------------------------------------------------------------
